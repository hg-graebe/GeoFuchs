\newcommand{\class}[1]{{\sf #1}}
\newcommand{\method}[1]{{\sf #1}}
\newcommand{\variable}[1]{{\sf #1}}
\newcommand{\ind}[1]{#1\index{#1}}


GMainFrame:

The \class{GMainFrame} main frame class is the entry point of the program. As
GUI it is composed in the usual way from different tool and status bars, a
tree panel (HGG: removed for the moment), and a main area, the {\tt desktop}
that holds the different GeoFuchs \ind{view}s as instances of
\class{GChildFrame}, a \class{JInternalFrame} subclass. The {\em current view}
is determined by the \method{desktop.getSelectedFrame()} method.

Each view is a \class{InternalFrame} that hosts a \class{GCoordinateArea} as
the main drawing area and a toolbar with some buttons. It knows about the
associated \ind{construction} (\method{theFrame.getConstruction()}) and
handles all actions (coordinate axes, rescaling) that are concerned with that
specific view only (view toolbar and {\em View pull-down menu} (Ansicht) of
the main frame). They are directly passed to the drawing area for processing.
Moving and rescaling the view does not affect the underlying construction.

The view's \class{GCoordinateArea} hosts additionally an affine transformation
matrix between the (mouse) coordinates of the view and the (world) coordinates
of the construction.  It handles all mouse and keyboard based events
and passes them to the underlying construction. \class{GInteractionArea} can
be removed, since it is only used in \class{GCoordinateArea}. 

ToDo: Better to understand the event handling mechanism. My idea: GMainFrame
knows desktop, desktop knows all GChildFrames, each GChildFrame knows its
GeoConstruction. GMainFrame handles all file menu and settings actions (set
the globally defined point and line colors for the next GeoObjects to be
created etc.)  Pushing other buttons in the GMainFrame changes the {\bf state}
according to that all actions within the currentFrame are interpreted.
Possible states are:
\begin{itemize}
\item (Display Change Action) Select GeoObjects in the currentConstruction to
  change their display options: get selection from currentView, open a
  dialogue menu as in ZuL, update options, update (through desktop) all
  related GChildFrames.
\item (Move Action) Select a GeoObject in the currentConstruction and move it:
  get selection from currentView, listen to MouseEvents, pass information to
  currentConstruction, update (through desktop) all related GChildFrames.
\item (Construction Step Action) Select a GeoTool and start a construction
  step: create a new GeoTool instance that controls the execution, i.e.,
  watches MouseEvents from currentView, gets desired selections from
  currentView, starts preview generation (this is tool specific!) via
  currentConstruction and, if all stuff is collected, updates
  currentConstruction and (through desktop) all related GChildFrames. This
  state can be interrupted, postponed (e.g., for moving some GeoObjects to a
  more convenient position) and restarted at the interruption point later on
  as long as the currentView does not change.
\end{itemize}
The description of the third item shows that much stuff in
\class{GCoordinateArea} should be refactored across these lines.  This should
be done through the newly added classes \class{GToolkit} (a hook for a more
sophisticated scan for all available GeoTools that contains -- for the moment
-- a list of all refactored GeoTools and a \method{createMenu(GMainFrame)}
method that returns a \class{JMenu} with all GeoTools to be incorporated into
the \class{GMainFrame}) and \class{ToolAction} that sets the
\variable{mainFrame.currentTool} to \variable{this.tool}.

Some more notes on the way, not yet serious:

A construction can have different views. Opening a construction a single
standard view is created. New views are created cloning currentView.  The tree
panel shows the \ind{tree view} of the currentConstruction (switched off for
the moment -- HGG).

A construction is a DAG of \class{GeoObjects} (g.o.) compiled step by step
applying \ind{construction tools} (c.t.) to create new g.o.s from the already
existing ones. The latter are the \ind{anchestors} of the new g.o.  Each g.o.\ 
is described by its coordinates that are computed via the c.t.\ algorithm
(c.t.a.) from the g.o.\ coordinates of the anchestors and the {\em display
  properties} (color, name, display mode etc.).

The display state of a new g.o.\ is determined by {\em global settings}
managed via the {\em state pull-down menu}.  It can be changed in a Display
Change Action. 

The available c.t.s are collected in the {\em tools pull-down menu}.  For easy
extension they are collected in a \class{GToolkit} object that is provided to
\class{GMainFrame} as parameter. 


